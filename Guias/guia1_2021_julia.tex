\documentclass[a4paper]{article}
\usepackage[latin1]{inputenc}
\usepackage{amsmath}
%\usepackage{babel}
\usepackage{hyperref} 

\topmargin -2truecm
\textheight 24truecm
\textwidth 17.5truecm
\oddsidemargin -0.5truecm
\parskip 1truemm
%\parsep 1truemm
\parindent 0em
\usepackage{amssymb}

\def\d{\,{\rm d}}

\def\RR{{\mathbb R}}
\def\CC{{\mathbb C}}
\def\NN{{\mathbb N}}
\def\ZZ{{\mathbb Z}}
\def\EE{{\mathbb E}}
\newcounter{problema}
\newcommand{\prob}[1]{\vspace{0.33cm}\stepcounter{problema}
                 \noindent{\bf Ejercicio \arabic{problema}:}{\it #1}}

\begin{document}
\renewcommand{\labelenumi}{\bf \alph{enumi})}
\renewcommand{\labelenumii}{\alph{enumi}$_\arabic{enumii}$)}
%\thispagestyle{empty}
\begin{center}
{\large {\bf M\'etodos Num\'ericos}}\\
{\small Aula Virtual: 
\href{https://famaf.aulavirtual.unc.edu.ar/course/view.php?id=704}
{https://famaf.aulavirtual.unc.edu.ar/course/view.php?id=704}
	} \\

\vspace*{1.5cm} 
\rule{0em}{2.em}
{\bf Gu\'{i}a 1} \\ 
{\em Julia} \\
Marzo de 2021
\end{center}
\vspace*{.5cm} 

% \noindent
% {\bf Nota:} La informaci\'on, en la computadora que use, es guardada de forma permanente, en lo que se suele llamar disco r\'igido.
% Es su responsabilidad mantener un sistema ordenado de trabajo, donde la informaci\'on est\'a guardada en un \'arbol natural de organizaci\'on.
% Dependiendo del sistema operativo, cuando se inicia una ventana de trabajo o terminal, se puede estar en la ubicaci\'on inicial, que gen\'ericamente llamaremos \emph{home}, que por ejemplo podr\'ia ser:

% \verb+    C:\usuarios\jaimito+

% si se est\'a en Windows, o por ejemplo en:

% \verb+    /home/jaimito+

% si se est\'a en linux (ubuntu).
% En el lugar donde en el ejemplo aparece {\tt jaimito}
% deber\'ia aparecer su nombre de usuario.
% De aqu\'i en adelante asumimos que trabaja con un sistema operativo linux,
% como por ejemplo \emph{ubuntu}.

\subsection*{Mini tutorial: uso de comandos de terminal}

Para usuarios Linux pueden ver una hoja de los comandos aqu\'i: 

\verb+    +\href{https://cheatography.com/jonathan992/cheat-sheets/gnu-linux-command-spanish/}
{https://cheatography.com/jonathan992/cheat-sheets/gnu-linux-command-spanish/}
    
Para usuarios de Windows ver:

\verb+    +\href{XXX}
{XXX}

\subsection*{Mini tutorial: gr\'aficos}


Tenga a mano el siguiente tutorial de la librer\'ia \verb+Plots+: 

\verb+    +\href{http://docs.juliaplots.org/latest/}
{http://docs.juliaplots.org/latest/}

El mecanismo b\'asico de ploteo toma dos vectores, $(x, y)$ y genera una serie de puntos en el plano con coordenadas los pares ordenados de los vectores 
$(x[i],y[i])$ si se usa la funci\'on scatter se plotean los puntos, si se usa la funci\'on \verb+plot+, 
estos puntos son interpolados por una curva. En algunos casos el vector $x$ puede ser elegido, por simplicidad, como un intervalo: \verb+x = 1:N+, o si queremos que tenga cierto intervalo predeterminado con \verb+x = 0:0.1:2+ por ejemplo. 
En Julia se puede 
aplicar una funci\'on definida para n\'umeros reales a un vector y esta retorna un vector con los valores de $f$ en los elementos del vector. 
Por lo tanto se puede usar la asignaci\'on: \verb+y = f.(x)+.
Note el \verb+.+ luego del nombre de la funci\'on, eso indica en el lenguaje que la funci\'on se aplica a cada uno de los elementos del vector.

Aqu\'i va un ejemplo. 
Llamo a la librer\'ia, de ese modo todos las funciones de la misma quedan autom\'aticamente accesibles.

\verb+    using Plots+

Defino los valores \verb+x+ de 4 en 4 desde -400 a 400

\verb+    x = -400:4:400+

Calculo los valores \verb+y+

\verb+    y = log.(10.0.^(x))+

Note los puntos, indican a Julia que cada funcion toma un vector de valores. 
Luego llamamos a la funci\'on scatter que literalmente graficar\'a los datos

\verb+    scatter(x,y)+

El gr\'afico aparecer\'a cortado pues la imagen de la funci\'on interna (potencia de 10) excede el intervalo de definicion de los numeros en \verb+Float64+.


\subsection*{Mini tutorial: vectores y matrices}

Veamos como declarar arreglos (del ingl\'es array). 
Consideremos un arreglo \verb+a+ declarado de la siguiente manera:

\verb+    a = Array{Float64,2}(undef,(50,20))+
   
Aqu\'i \verb+Array+ es el nombre de una funci\'on. 
Las expresi\'on en llaves es de la forma \verb+{T,N}+ indicando el tipo de los elementos (en este caso, n\'umeros de punto flotante de 64 bits) y el n\'umero de dimensiones (2 en este caso, por lo que el arreglo representa una matriz). 
Finalmente, entre par\'entesis, la primera expresi\'on indica especifica como se han de inicializar los elementos del array (en este caso se usa \verb+undef+, lo cual indica que se dejen los elementos sin inicializar), la segunda expresi\'on indica el tama\~no de las dimensiones del array (en este caso se usa la tupla de dos elementos \verb+(50,20)+, lo cual indica que la primera dimensi\'on es de tama\~no 50 y la segunda de tama\~no 20). 

Un arreglo con s\'olo 1 dimensi\'on es considerado un vector. 
Por ende, un vector \verb+v+ pude declarase de la forma

\verb+    v = Array{Float64,1}(undef,30)+
    
En este caso, no hace falta especificar el tama\~no de la \'unica dimensi\'on usando una tupla. 
Con un simple n\'umero es suficiente. 
Alternativamente, un vector puede declarase de la forma

\verb+    v = Vector{Float64}(undef,30)+
    
Internamente, Julia remplazar\'a esta \'ultima expresi\'on por la anterior. 
Finalmente, es posible declarar vectores de tama\~no nulo, cuyos tama\~nos pueden ser posteriormente modificados. 
Por ejemplo, a expresi\'on

\verb+    a = Array{Float64,1}()+
    
declara un arreglo de 1 dimensi\'on de tama\~no 0. 
El mismo puede ir rellenandose, por ejemplo, usando la funcion \verb+push!+ (recordemos que las funciones que terminan con el s\'imbolo \verb+!+ sirven para modificar contenido). 
Por ejemplo, si escribimos y ejecutamos el c\'odigo:

\begin{verbatim}
    a = Array{Float64,1}()
    println(a)
    push!(a,10.0)
    push!(a,20.0)
    push!(a,30.0)
    println(a)
\end{verbatim}
    
Veremos como resultado las expresiones:

\begin{verbatim}
    Float64[]
    [10.0, 20.0, 30.0]
\end{verbatim}    
    
correspondiendo, primero a un arreglo vac\'io y luego a un arreglo con 3 elementos. 
Para mayor informaci\'on sobre la declaraci\'on de arreglos, visite la p\'agina:

\verb+    +\href{https://docs.julialang.org/en/v1/base/arrays/}
{https://docs.julialang.org/en/v1/base/arrays/}  

\subsection*{Ejercicios}

\prob{} 
En una terminal, cree un directorio nuevo usando el comando \verb+mkdir+. 
Descargue desde el navegador el archivo de esta gu\'ia y mueva el mismo al nuevo directorio.
Cambie el nombre del archivo usando \verb+mv+ a {\em guia1.pdf} y copie usando \verb+cp+ el archivo a otro con nombre {\em Copia.pdf}.
Luego, liste usando \verb+ls+ el contenido del directorio y verifique los nombres y el tama\~no y fecha de los mismos usando \verb+ls -l+.
Averig\"ue qu\'e espacio ocupa el directorio usando el comando \verb+du -h+.
Finalmente, borre todos los archivos y el directorio usando \verb+rm+ y \verb+rmdir+.

\prob{} 
Grafique la funci\'on $f(x)=\sin(2\pi x)$ para $x\in[0,5]$. Var\'ie la cantidad de puntos que usa para mejorar la calidad del gr\'afico. De nombre a los ejes, t\'itulo al gr\'afico y defina la leyenda para que aparezca $f(x)$.
Por regla general, use la funci\'on \verb+scatter+ para graficar puntos y la funci\'on \verb+plot+ para graficar curvas. 
Tenga a mano el tutorial de la librer\'ia o paquete \verb+Plots+:

\verb+    +\href{http://docs.juliaplots.org/latest/}
{http://docs.juliaplots.org/latest/}

\prob{} 
\begin{itemize}
\item[a)] 
Cree un directorio nuevo, \verb+mkdir tmp/+. 
Ingrese al nuevo directorio, \verb+cd tmp+.

\item[b)] 
Dentro del directorio creado, use su editor de texto preferido o las funciones \verb+open+ y \verb+print+ de Julia para crear un archivo de texto llamado \verb+datos.dat+ con el siguiente contenido

\verb+    0.0  1.0  1.2+

\verb+    1.0  1.5  1.6+

\verb+    2.0  2.0  2.0+

\verb+    3.6  2.5  2.4+

\item[c)] 
Usar las funciones \verb+Vector+ para crear arrays llamados \verb+xvals+, \verb+y1vals+ e \verb+y2vals+. Luego utilice las funciones \verb+open+, \verb+eachline+, \verb+parse+ y \verb+push!+ para cargar los datos de las columnas del archivo \verb+datos.dat+ en dichos arrays.

\item[d)] 
Utilice la funci\'on \verb+scatter+ del paquete \verb+Plots+ para graficar las funciones definidas por \verb+y1vals+ vs \verb+xvals+ e \verb+y2vals+ vs \verb+xvals+.

\item[e)] 
En el mismo gr\'afico incluya las funciones $y_3(x) = 0.5x + 1$ e $y_4(x) = 0.4x + 1.2$

\item[f)] 
Agregue el labels $x$ al eje horizontal y el label $y$ al vertical.

\item[g)] 
Titule el gr\'afico ``Funciones lineales''.

\item[h)] 
Pruebe habilitar y deshabilitar la grilla.

\item[i)] 
Exporte el gr\'afico a un archivo formato \verb+.pdf+ (u otro formato de su preferencia, tales como \verb+.png+, \verb+.jpg+ o \verb+.ps+). 
Utilice su visualizador de \verb+.pdf+ (o \verb+.jpg+, etc) preferido para constatar los cambios.

\item[j)] 
Experimente graficar cambiando el tama\~no de los puntos y el grosor de las l\'ineas, etc.

\item[k)] 
Modifique el archivo \verb+datos.dat+ agregando, sacando y/o cambiando n\'umeros. 
Luego regrafique y constante los cambios.
\end{itemize}

{\bf Ejercicio 3 bis:}
En Julia existe varias formas alternativas y convenientes de guardar en, y leer datos de, archivos. Una forma la provee la librer\'ia o paquete \verb+CSV+, que aqu\'i no veremos. 
Otra particularmente conveniente utiliza el formato \verb+JLD2+ y es prove\'ida por el paquete \verb+FileIO+. 
Ejercit\'emonos con un ejemplo.

\begin{itemize}
\item[a)]
Guarde los arreglos \verb+vals+, \verb+y1vals+ e \verb+y2vals+ en un archivo usando el formato \verb+JLD2+ v\'ia el c\'odigo

\begin{verbatim}
    mkdir("tmp")
    save("tmp/datos.jld2", Dict("xvals" => xvals, "y1vals" => y1vals))    
\end{verbatim}

\item[b)]
Cargue dichos arreglos del archivo \verb+datos.jld2+ a nuevos arreglos de nombres \verb+xnuevo+, \verb+y1nuevo+ e \verb+y2nuevo+ v\'ia el c\'odigo

\begin{verbatim}
    xnuevo = load("tmp/datos.jld2", "xvals")
    y1nuevo = load("tmp/datos.jld2", "y1vals")
    y2nuevo = load("tmp/datos.jld2", "y1vals")
\end{verbatim}
    
\item[c)]
Grafique \verb+y1nuevo+ vs \verb+xnuevo+ e \verb+y2nuevo+ vs \verb+xnuevo+. 

\item[d)]
Finalmente, borre el directorio temporario escribiendo

\verb+    rm("tmp")+
\end{itemize}

\prob{}
Evaluar en celdas de Julia las siguientes operaciones matem\'aticas.

Usando divisi\'on entera eval\'ue:

\begin{itemize}
\item[a)]
$5 / 2 + 20/ 6$

\item[b)]
b) $4 \times 6 / 2 - 15 / 2$

\item[c)]
$5 \times 15 / 2 / (4 - 2)$
\end{itemize}

Usando divisi\'on entera, flotante y racional eval\'ue:

\begin{itemize}
\item[d)]
$1 + 1/4$

\item[e)]
$5.7/3$
\end{itemize}

Ayudas:

La divisi\'on entera se representa con el s\'imbolo unicode $\div$.
Para escribir o ingresar dicho s\'imbolo en la c\'onsola de Julia o en una notebook de Julia, escribir \verb+\div+ e inmediatamente despu\'es presionar la tecla \verb+tab+.
De manera an\'aloga se escribe cualquier otro s\'imbolo unicode.
En la siguiente p\'agina Ud. puede encontrar algunos s\'imbolos unicode relevantes para Julia.

\verb+    +\href{https://docs.julialang.org/en/v1/manual/unicode-input/}
{https://docs.julialang.org/en/v1/manual/unicode-input/}

La divisi\'on flotante se representa con la barra \verb+/+.
La divisi\'on racional se representa con la doble barra `//`.

Para crear un entero de un dado n\'umero de bits usamos funciones como \verb+Int16()+, \verb+Int32()+, etc. 
Para crear n\'umeros de punto flotante usamos \verb+Float32()+, \verb+Float64()+, etc. 
Tambi\'en podemos crear enteros sin signo usando \verb+UInt16()+, \verb+UInt32()+, etc.
Cuando el tipo de un n\'umero tipo no se especifica expl\'icitamente, Julia asigna por defecto el tipo \verb+Float64+ o el tipo \verb+Int64+ dependiendo de si el n\'umero tiene punto o no, o dependiendo del tipo de operaci\'on involucrada (ej. divisi\'on en punto flotante, entera o racional).
Para m\'as detalles visite la p\'agina:

\verb+    +\href{https://docs.julialang.org/en/v1/manual/integers-and-floating-point-numbers/}
{https://docs.julialang.org/en/v1/manual/integers-and-floating-point-numbers/}

\prob{}   
Evaluar las siguientes operaciones usando matem\'atica de enteros con signo de 2 bytes (\verb+Int16+ del ingl\'es Integer de 16 bits):

\begin{itemize}
\item[a)] 
$32767 + 1$

\item[b)] 
$30000 \times 2$

\item[c)] 
$-30000-10000$
\end{itemize}

y usando enteros sin signo de 2 bytes (\verb+UInt16+ del ingl\'es Unsigned Integer de 16 bits):

\begin{itemize}
\item[d)] 
$32767 + 1$

\item[e)] 
$64535 + 1$
\end{itemize}

\prob{} 
Escriba un programa que pida dos n\'umeros reales e imprima en la pantalla el mayor de ellos. El programa debe indicar si los n\'umeros son iguales.

% 3
\prob{} 
Escriba un programa que pida un n\'umero entero y determine si es m\'ultiplo de 2 y de 5.

% 4
\prob{} 
Escriba una programa que ingrese los coeficientes $A$, $B$ y $C$ de un polinomio real de segundo grado, $Ax^2+Bx+C$, calcule e imprima en pantalla las dos ra\'ices del polinomio en formato complejo $a+ib$, sin utilizar \'algebra compleja.

\prob{} 
Escriba un programa que permita convertir n\'umeros naturales con base 10 a la base $b \leq 16$.
El programa debe pedir como entrada $b$ y el n\'umero natural a convertir (que debe estar en base 10).
Ayuda: desarrolle el algoritmo usando lápiz y papel.

% 5
%\prob{} Dise\~nar una funci\'on que calcule la potencia en\'esima de un n\'umero, es
%decir que devuelva $X^n$ para $X$ real y $n$ entero. Realice un programa que
%utilice la funci\'on e imprima en pantalla las primeras 5 potencias naturales de
%un n\'umero ingresado.


%\prob{} Los per\'{\i}metros $\pi_k$ de poligonos regulares de $k$ lados inscriptos en un 
%c\'{\i}rculo de diametro unidad son $\pi_8=3.061467,\,\pi_{16}=3.121445,\,\pi_{32}=3.136548,\,\pi_{64}=3.140331$ 
%respectivamente, mientras que los per\'{\i}metros de poligonos regulares circunscriptos% son 
% $\Pi_8=3.313708,\,\Pi_{16}=3.182598,\,\Pi_{32}=3.151725,\,\Pi_{64}=3.144118$
%respectivamente.  Asuma que 
%
%\[
%\pi_k\,=\,c_0+\frac{c_1}{k}+\frac{c_2}{k^2}+\frac{c_3}{k^3}+\ldots \;\;\;;\;\;\;
%\Pi_k\,=\,C_0+\frac{C_1}{k}+\frac{C_2}{k^2}+\frac{C_3}{k^3}+\ldots
%\]
%
%De una estimaci\'on para $\pi$.
%Compare este resultado con la mejor evaluaci\'on que se puede obtener con el algoritmo %de 
%Arq\'{\i}medes.

% 7
\prob{}
Escriba un programa para calcular un valor aproximado de $\pi$ utilizando:

\begin{itemize}
\item[a)]
la f\'ormula recurrente de Arqu\'imedes, en dode los par\'ametros $p_n$ y $P_n$ de los pol\'igonos regulares de $n$ lados inscriptos y circunscriptos en la circunsferencia de radio $1/2$, respectivamente, se usan para acotar $\pi$ ya que $p_n < \pi < P_n$ para todo $n$ y $P_n-p_n$ disminuye con $n$.

La f\'ormula de recurrencia que encontr\'o Arqu\'imides es la siguiente:

$$P_{2n} = \frac{2p_nP_n}{p_n + P_n} \;\;\;\;\; p_{2n} = \sqrt{P_{2n}p_n}$$

Usando los valores $P_6 = 2\sqrt{3}$ y $p_6 = 3$, correspondientes al hex\'agono, escriba un programa que realice 20 iteraciones, con $n = 6 \cdot 2^k$, y $k = 1 \ldots 20$, y escriba los resultados en pantalla.

\item[b)]
la productoria de Wallis
$$\frac{\pi}{2} = \prod_1^{\infty} \frac{(2n)^2}{(2n)^2 - 1} = \frac{3}{4}\frac{16}{15}\frac{36}{35}\frac{64}{63} \cdots$$

Calcule el valor de $\pi$ truncando la productoria a $10^6$ factores.
\end{itemize}    

%%%%%%%%%%  comienzo guia_c
\prob{} 
Escriba un vector cuyas $N$ componentes satisfagan $v[i] = i^2$.

\prob{} 
Declare un arreglo de n\'umeros flotantes de 64 bits de dimensiones \verb+(50,20)+ en donde la entrada \verb+a[i,j]+ de la fila \verb+i+ y la columna \verb+j+ de \verb+a+ valga \verb+i*j+.
Escriba secciones de este arreglo representando:
    
\begin{itemize}
\item[a)]
la primera fila de \verb+a+,

\item[b)]
la \'ultima columna de \verb+a+, y

\item[c)]
un elemento de por medio en cada fila y columna de \verb+a+.
\end{itemize}

\prob{} 
Escriba un programa, utilizando la instrucci\'on \verb+for+, que multiplique un vector de $N$ elementos, por una matriz de $N \times N$. 
El programa debe preguntar el valor de $N$ y luego definir los arreglos, y darle valores iniciales tal que, la matriz sea triangular superior, con todos sus elementos igual a 1, excepto los de la diagonal que toman valor 3, el vector tendra todos sus elementos pares igual a 2, y los impares igual a 3. 
No utilice \verb+do+ ni \verb+for+ para inicializar el vector, ni \verb+do+ ni \verb+for+ anidados para inicializar la matriz.
Finalmente, calcule el producto matricial de la matriz por el vector e impr\'imalos en pantalla.

\bigskip
\bigskip
\newpage
\subsection*{Ejercicios Complementarios}

\prob{} 
Un script de Julia es un archivo que contiene c\'odigo Julia. 

\begin{itemize}
\item[a)]
Cree un archivo llamado \verb+miscript.jl+ con el siguiente contenido

\begin{verbatim}
    function maximo(x,y)
        if x>y
            return x
        else
            return y
        end
    end
\end{verbatim}

\item[b)]
En una celda de la notebook, ejecute las lineas

\begin{verbatim}
    include("miscript.jl")
    println(maximo(1,13))
\end{verbatim}    
    
para incluir y usar las funci\'on declarada en el script en la actual sesi\'on de la notebook.

\item[c)]
Experimente agregando otras funciones al script. Vuelva a incluir el escript y pruebe si las funciones pueden usarse en la notebook.
\end{itemize}

\prob{} 
\begin{itemize}
\item[a)] 
Escriba un programa para calcular la posici\'on y la velocidad en funci\'on del tiempo, para un problema de tiro obl\'icuo. 
Asumiendo que el proyectil parte del origen, debe preguntar el \'angulo (en grados) y la velocidad inicial (en m/seg.). 
Utilice funciones para calcular la posici\'on y la velocidad.

\item[b)] 
Grafique la soluci\'on (trayectoria) eligiendo el incremento temporal $\Delta t$ de manera que la gr\'afica tenga 600 puntos y abarque el intervalo entre el disparo y el instante en que el proyectil vuelve a tener altura 0. 

\item[c)] 
Grafique las componentes de la velocidad en funci\'on del tiempo. 
\end{itemize}

% 6
\prob{} 
Se pretende calcular las sumas $S_N = \sum^N_{k=1} a_k$, donde $N$ es un número natural. 
Llamemos $S'_N$ al valor num\'ericamente calculado que se logra de hacer $(S_{N-1} + a_N)$ utilizando n\'umeros de punto flotante de 32 bits.
Sea $S_N =\sum^N_{k=1} \frac{1}{k}$. 
Mostrar que $S'_N$ se estaciona a partir de alg\'un $N$ suficientemente grande. 
Deducir que a partir de entonces $S_N \neq S'_{N}$. Hacer un programa que determine el valor a partir del cual $S'_N$ se estaciona.

\prob{} 
Para calcular un valor aproximado de $\pi$ utilizaremos la siguiente serie infinita alternante:

$$\frac{\pi}{4} = \sum_{n=0}^{\infty} \frac{(-1)^n}{2n+1}$$

Recordemos que una cota superior para el error cometido al truncar una serie alternante (de valor absoluto decreciente) est\'a dado por el valor absoluto del primer t\'ermino despreciado. 
Escriba un programa que ingrese el n\'umero de cifras decimales exactas con que se desea el valor de $\pi$ (entre 1 y 5 cifras) y devuelva en pantalla el n\'umero de t\'erminos que deben incluirse en la serie de arriba para obtener dicha precisi\'on y a rengl\'on siguiente el valor de $\pi$ obtenido de esta forma, truncado el resultado al n\'umero de cifras pedido.

% 
\prob{} 
Escribir un programa que pida una contrase\~na de tres d\'igitos y permita
leer tres intentos.  Si el usuario da la contrase\~na correcta responde
``Correcto'' y queda inactivo, con este mensaje.  En caso contrario el programa
escribe ``Lo siento, contrase\~na equivocada'' y se cierra de inmediato.

\prob{} 
Escribir un programa que, dado un a\~no y el nombre de un mes, {\bf saque por pantalla} el n\'umero de d\'ias del mes (tenga en cuenta que algunos a\~nos son bisiestos).

\prob{} 
Escriba un programa que genere secuencialmente 10 archivos con nombre de salida diferentes (dependiendo del valor que tome alg\'un par\'ametro). 
En cada archivo, escriba bajo la forma de dupla 

\verb+    x f(x)+
    
una funci\'on evaluada en $N$ puntos y que tambi\'en dependa del par\'ametro (por ejemplo $y = sin(\pi \omega x)$, con $\omega = 1, 2, 3, \ldots$). 
El loop debe cerrar cada archivo luego de escribir en \'el.

En el mismo programa o en otro, construya otro loop que abra secuencialmente los archivos y que (sin borrar los datos escritos) agregue otros $N$ pares de duplas de la funci\'on.

\end{document}
