
% Default to the notebook output style

    


% Inherit from the specified cell style.




    
\documentclass[11pt]{article}

    
    
    \usepackage[T1]{fontenc}
    % Nicer default font (+ math font) than Computer Modern for most use cases
    \usepackage{mathpazo}

    % Basic figure setup, for now with no caption control since it's done
    % automatically by Pandoc (which extracts ![](path) syntax from Markdown).
    \usepackage{graphicx}
    % We will generate all images so they have a width \maxwidth. This means
    % that they will get their normal width if they fit onto the page, but
    % are scaled down if they would overflow the margins.
    \makeatletter
    \def\maxwidth{\ifdim\Gin@nat@width>\linewidth\linewidth
    \else\Gin@nat@width\fi}
    \makeatother
    \let\Oldincludegraphics\includegraphics
    % Set max figure width to be 80% of text width, for now hardcoded.
    \renewcommand{\includegraphics}[1]{\Oldincludegraphics[width=.8\maxwidth]{#1}}
    % Ensure that by default, figures have no caption (until we provide a
    % proper Figure object with a Caption API and a way to capture that
    % in the conversion process - todo).
    \usepackage{caption}
    \DeclareCaptionLabelFormat{nolabel}{}
    \captionsetup{labelformat=nolabel}

    \usepackage{adjustbox} % Used to constrain images to a maximum size 
    \usepackage{xcolor} % Allow colors to be defined
    \usepackage{enumerate} % Needed for markdown enumerations to work
    \usepackage{geometry} % Used to adjust the document margins
    \usepackage{amsmath} % Equations
    \usepackage{amssymb} % Equations
    \usepackage{textcomp} % defines textquotesingle
    % Hack from http://tex.stackexchange.com/a/47451/13684:
    \AtBeginDocument{%
        \def\PYZsq{\textquotesingle}% Upright quotes in Pygmentized code
    }
    \usepackage{upquote} % Upright quotes for verbatim code
    \usepackage{eurosym} % defines \euro
    \usepackage[mathletters]{ucs} % Extended unicode (utf-8) support
    \usepackage[utf8x]{inputenc} % Allow utf-8 characters in the tex document
    \usepackage{fancyvrb} % verbatim replacement that allows latex
    \usepackage{grffile} % extends the file name processing of package graphics 
                         % to support a larger range 
    % The hyperref package gives us a pdf with properly built
    % internal navigation ('pdf bookmarks' for the table of contents,
    % internal cross-reference links, web links for URLs, etc.)
    \usepackage{hyperref}
    \usepackage{longtable} % longtable support required by pandoc >1.10
    \usepackage{booktabs}  % table support for pandoc > 1.12.2
    \usepackage[inline]{enumitem} % IRkernel/repr support (it uses the enumerate* environment)
    \usepackage[normalem]{ulem} % ulem is needed to support strikethroughs (\sout)
                                % normalem makes italics be italics, not underlines
    

    
    
    % Colors for the hyperref package
    \definecolor{urlcolor}{rgb}{0,.145,.698}
    \definecolor{linkcolor}{rgb}{.71,0.21,0.01}
    \definecolor{citecolor}{rgb}{.12,.54,.11}

    % ANSI colors
    \definecolor{ansi-black}{HTML}{3E424D}
    \definecolor{ansi-black-intense}{HTML}{282C36}
    \definecolor{ansi-red}{HTML}{E75C58}
    \definecolor{ansi-red-intense}{HTML}{B22B31}
    \definecolor{ansi-green}{HTML}{00A250}
    \definecolor{ansi-green-intense}{HTML}{007427}
    \definecolor{ansi-yellow}{HTML}{DDB62B}
    \definecolor{ansi-yellow-intense}{HTML}{B27D12}
    \definecolor{ansi-blue}{HTML}{208FFB}
    \definecolor{ansi-blue-intense}{HTML}{0065CA}
    \definecolor{ansi-magenta}{HTML}{D160C4}
    \definecolor{ansi-magenta-intense}{HTML}{A03196}
    \definecolor{ansi-cyan}{HTML}{60C6C8}
    \definecolor{ansi-cyan-intense}{HTML}{258F8F}
    \definecolor{ansi-white}{HTML}{C5C1B4}
    \definecolor{ansi-white-intense}{HTML}{A1A6B2}

    % commands and environments needed by pandoc snippets
    % extracted from the output of `pandoc -s`
    \providecommand{\tightlist}{%
      \setlength{\itemsep}{0pt}\setlength{\parskip}{0pt}}
    \DefineVerbatimEnvironment{Highlighting}{Verbatim}{commandchars=\\\{\}}
    % Add ',fontsize=\small' for more characters per line
    \newenvironment{Shaded}{}{}
    \newcommand{\KeywordTok}[1]{\textcolor[rgb]{0.00,0.44,0.13}{\textbf{{#1}}}}
    \newcommand{\DataTypeTok}[1]{\textcolor[rgb]{0.56,0.13,0.00}{{#1}}}
    \newcommand{\DecValTok}[1]{\textcolor[rgb]{0.25,0.63,0.44}{{#1}}}
    \newcommand{\BaseNTok}[1]{\textcolor[rgb]{0.25,0.63,0.44}{{#1}}}
    \newcommand{\FloatTok}[1]{\textcolor[rgb]{0.25,0.63,0.44}{{#1}}}
    \newcommand{\CharTok}[1]{\textcolor[rgb]{0.25,0.44,0.63}{{#1}}}
    \newcommand{\StringTok}[1]{\textcolor[rgb]{0.25,0.44,0.63}{{#1}}}
    \newcommand{\CommentTok}[1]{\textcolor[rgb]{0.38,0.63,0.69}{\textit{{#1}}}}
    \newcommand{\OtherTok}[1]{\textcolor[rgb]{0.00,0.44,0.13}{{#1}}}
    \newcommand{\AlertTok}[1]{\textcolor[rgb]{1.00,0.00,0.00}{\textbf{{#1}}}}
    \newcommand{\FunctionTok}[1]{\textcolor[rgb]{0.02,0.16,0.49}{{#1}}}
    \newcommand{\RegionMarkerTok}[1]{{#1}}
    \newcommand{\ErrorTok}[1]{\textcolor[rgb]{1.00,0.00,0.00}{\textbf{{#1}}}}
    \newcommand{\NormalTok}[1]{{#1}}
    
    % Additional commands for more recent versions of Pandoc
    \newcommand{\ConstantTok}[1]{\textcolor[rgb]{0.53,0.00,0.00}{{#1}}}
    \newcommand{\SpecialCharTok}[1]{\textcolor[rgb]{0.25,0.44,0.63}{{#1}}}
    \newcommand{\VerbatimStringTok}[1]{\textcolor[rgb]{0.25,0.44,0.63}{{#1}}}
    \newcommand{\SpecialStringTok}[1]{\textcolor[rgb]{0.73,0.40,0.53}{{#1}}}
    \newcommand{\ImportTok}[1]{{#1}}
    \newcommand{\DocumentationTok}[1]{\textcolor[rgb]{0.73,0.13,0.13}{\textit{{#1}}}}
    \newcommand{\AnnotationTok}[1]{\textcolor[rgb]{0.38,0.63,0.69}{\textbf{\textit{{#1}}}}}
    \newcommand{\CommentVarTok}[1]{\textcolor[rgb]{0.38,0.63,0.69}{\textbf{\textit{{#1}}}}}
    \newcommand{\VariableTok}[1]{\textcolor[rgb]{0.10,0.09,0.49}{{#1}}}
    \newcommand{\ControlFlowTok}[1]{\textcolor[rgb]{0.00,0.44,0.13}{\textbf{{#1}}}}
    \newcommand{\OperatorTok}[1]{\textcolor[rgb]{0.40,0.40,0.40}{{#1}}}
    \newcommand{\BuiltInTok}[1]{{#1}}
    \newcommand{\ExtensionTok}[1]{{#1}}
    \newcommand{\PreprocessorTok}[1]{\textcolor[rgb]{0.74,0.48,0.00}{{#1}}}
    \newcommand{\AttributeTok}[1]{\textcolor[rgb]{0.49,0.56,0.16}{{#1}}}
    \newcommand{\InformationTok}[1]{\textcolor[rgb]{0.38,0.63,0.69}{\textbf{\textit{{#1}}}}}
    \newcommand{\WarningTok}[1]{\textcolor[rgb]{0.38,0.63,0.69}{\textbf{\textit{{#1}}}}}
    
    
    % Define a nice break command that doesn't care if a line doesn't already
    % exist.
    \def\br{\hspace*{\fill} \\* }
    % Math Jax compatability definitions
    \def\gt{>}
    \def\lt{<}
    % Document parameters
    \title{Guía 7, 2021, Julia}
    
    
    

    % Pygments definitions
    
\makeatletter
\def\PY@reset{\let\PY@it=\relax \let\PY@bf=\relax%
    \let\PY@ul=\relax \let\PY@tc=\relax%
    \let\PY@bc=\relax \let\PY@ff=\relax}
\def\PY@tok#1{\csname PY@tok@#1\endcsname}
\def\PY@toks#1+{\ifx\relax#1\empty\else%
    \PY@tok{#1}\expandafter\PY@toks\fi}
\def\PY@do#1{\PY@bc{\PY@tc{\PY@ul{%
    \PY@it{\PY@bf{\PY@ff{#1}}}}}}}
\def\PY#1#2{\PY@reset\PY@toks#1+\relax+\PY@do{#2}}

\expandafter\def\csname PY@tok@w\endcsname{\def\PY@tc##1{\textcolor[rgb]{0.73,0.73,0.73}{##1}}}
\expandafter\def\csname PY@tok@c\endcsname{\let\PY@it=\textit\def\PY@tc##1{\textcolor[rgb]{0.25,0.50,0.50}{##1}}}
\expandafter\def\csname PY@tok@cp\endcsname{\def\PY@tc##1{\textcolor[rgb]{0.74,0.48,0.00}{##1}}}
\expandafter\def\csname PY@tok@k\endcsname{\let\PY@bf=\textbf\def\PY@tc##1{\textcolor[rgb]{0.00,0.50,0.00}{##1}}}
\expandafter\def\csname PY@tok@kp\endcsname{\def\PY@tc##1{\textcolor[rgb]{0.00,0.50,0.00}{##1}}}
\expandafter\def\csname PY@tok@kt\endcsname{\def\PY@tc##1{\textcolor[rgb]{0.69,0.00,0.25}{##1}}}
\expandafter\def\csname PY@tok@o\endcsname{\def\PY@tc##1{\textcolor[rgb]{0.40,0.40,0.40}{##1}}}
\expandafter\def\csname PY@tok@ow\endcsname{\let\PY@bf=\textbf\def\PY@tc##1{\textcolor[rgb]{0.67,0.13,1.00}{##1}}}
\expandafter\def\csname PY@tok@nb\endcsname{\def\PY@tc##1{\textcolor[rgb]{0.00,0.50,0.00}{##1}}}
\expandafter\def\csname PY@tok@nf\endcsname{\def\PY@tc##1{\textcolor[rgb]{0.00,0.00,1.00}{##1}}}
\expandafter\def\csname PY@tok@nc\endcsname{\let\PY@bf=\textbf\def\PY@tc##1{\textcolor[rgb]{0.00,0.00,1.00}{##1}}}
\expandafter\def\csname PY@tok@nn\endcsname{\let\PY@bf=\textbf\def\PY@tc##1{\textcolor[rgb]{0.00,0.00,1.00}{##1}}}
\expandafter\def\csname PY@tok@ne\endcsname{\let\PY@bf=\textbf\def\PY@tc##1{\textcolor[rgb]{0.82,0.25,0.23}{##1}}}
\expandafter\def\csname PY@tok@nv\endcsname{\def\PY@tc##1{\textcolor[rgb]{0.10,0.09,0.49}{##1}}}
\expandafter\def\csname PY@tok@no\endcsname{\def\PY@tc##1{\textcolor[rgb]{0.53,0.00,0.00}{##1}}}
\expandafter\def\csname PY@tok@nl\endcsname{\def\PY@tc##1{\textcolor[rgb]{0.63,0.63,0.00}{##1}}}
\expandafter\def\csname PY@tok@ni\endcsname{\let\PY@bf=\textbf\def\PY@tc##1{\textcolor[rgb]{0.60,0.60,0.60}{##1}}}
\expandafter\def\csname PY@tok@na\endcsname{\def\PY@tc##1{\textcolor[rgb]{0.49,0.56,0.16}{##1}}}
\expandafter\def\csname PY@tok@nt\endcsname{\let\PY@bf=\textbf\def\PY@tc##1{\textcolor[rgb]{0.00,0.50,0.00}{##1}}}
\expandafter\def\csname PY@tok@nd\endcsname{\def\PY@tc##1{\textcolor[rgb]{0.67,0.13,1.00}{##1}}}
\expandafter\def\csname PY@tok@s\endcsname{\def\PY@tc##1{\textcolor[rgb]{0.73,0.13,0.13}{##1}}}
\expandafter\def\csname PY@tok@sd\endcsname{\let\PY@it=\textit\def\PY@tc##1{\textcolor[rgb]{0.73,0.13,0.13}{##1}}}
\expandafter\def\csname PY@tok@si\endcsname{\let\PY@bf=\textbf\def\PY@tc##1{\textcolor[rgb]{0.73,0.40,0.53}{##1}}}
\expandafter\def\csname PY@tok@se\endcsname{\let\PY@bf=\textbf\def\PY@tc##1{\textcolor[rgb]{0.73,0.40,0.13}{##1}}}
\expandafter\def\csname PY@tok@sr\endcsname{\def\PY@tc##1{\textcolor[rgb]{0.73,0.40,0.53}{##1}}}
\expandafter\def\csname PY@tok@ss\endcsname{\def\PY@tc##1{\textcolor[rgb]{0.10,0.09,0.49}{##1}}}
\expandafter\def\csname PY@tok@sx\endcsname{\def\PY@tc##1{\textcolor[rgb]{0.00,0.50,0.00}{##1}}}
\expandafter\def\csname PY@tok@m\endcsname{\def\PY@tc##1{\textcolor[rgb]{0.40,0.40,0.40}{##1}}}
\expandafter\def\csname PY@tok@gh\endcsname{\let\PY@bf=\textbf\def\PY@tc##1{\textcolor[rgb]{0.00,0.00,0.50}{##1}}}
\expandafter\def\csname PY@tok@gu\endcsname{\let\PY@bf=\textbf\def\PY@tc##1{\textcolor[rgb]{0.50,0.00,0.50}{##1}}}
\expandafter\def\csname PY@tok@gd\endcsname{\def\PY@tc##1{\textcolor[rgb]{0.63,0.00,0.00}{##1}}}
\expandafter\def\csname PY@tok@gi\endcsname{\def\PY@tc##1{\textcolor[rgb]{0.00,0.63,0.00}{##1}}}
\expandafter\def\csname PY@tok@gr\endcsname{\def\PY@tc##1{\textcolor[rgb]{1.00,0.00,0.00}{##1}}}
\expandafter\def\csname PY@tok@ge\endcsname{\let\PY@it=\textit}
\expandafter\def\csname PY@tok@gs\endcsname{\let\PY@bf=\textbf}
\expandafter\def\csname PY@tok@gp\endcsname{\let\PY@bf=\textbf\def\PY@tc##1{\textcolor[rgb]{0.00,0.00,0.50}{##1}}}
\expandafter\def\csname PY@tok@go\endcsname{\def\PY@tc##1{\textcolor[rgb]{0.53,0.53,0.53}{##1}}}
\expandafter\def\csname PY@tok@gt\endcsname{\def\PY@tc##1{\textcolor[rgb]{0.00,0.27,0.87}{##1}}}
\expandafter\def\csname PY@tok@err\endcsname{\def\PY@bc##1{\setlength{\fboxsep}{0pt}\fcolorbox[rgb]{1.00,0.00,0.00}{1,1,1}{\strut ##1}}}
\expandafter\def\csname PY@tok@kc\endcsname{\let\PY@bf=\textbf\def\PY@tc##1{\textcolor[rgb]{0.00,0.50,0.00}{##1}}}
\expandafter\def\csname PY@tok@kd\endcsname{\let\PY@bf=\textbf\def\PY@tc##1{\textcolor[rgb]{0.00,0.50,0.00}{##1}}}
\expandafter\def\csname PY@tok@kn\endcsname{\let\PY@bf=\textbf\def\PY@tc##1{\textcolor[rgb]{0.00,0.50,0.00}{##1}}}
\expandafter\def\csname PY@tok@kr\endcsname{\let\PY@bf=\textbf\def\PY@tc##1{\textcolor[rgb]{0.00,0.50,0.00}{##1}}}
\expandafter\def\csname PY@tok@bp\endcsname{\def\PY@tc##1{\textcolor[rgb]{0.00,0.50,0.00}{##1}}}
\expandafter\def\csname PY@tok@fm\endcsname{\def\PY@tc##1{\textcolor[rgb]{0.00,0.00,1.00}{##1}}}
\expandafter\def\csname PY@tok@vc\endcsname{\def\PY@tc##1{\textcolor[rgb]{0.10,0.09,0.49}{##1}}}
\expandafter\def\csname PY@tok@vg\endcsname{\def\PY@tc##1{\textcolor[rgb]{0.10,0.09,0.49}{##1}}}
\expandafter\def\csname PY@tok@vi\endcsname{\def\PY@tc##1{\textcolor[rgb]{0.10,0.09,0.49}{##1}}}
\expandafter\def\csname PY@tok@vm\endcsname{\def\PY@tc##1{\textcolor[rgb]{0.10,0.09,0.49}{##1}}}
\expandafter\def\csname PY@tok@sa\endcsname{\def\PY@tc##1{\textcolor[rgb]{0.73,0.13,0.13}{##1}}}
\expandafter\def\csname PY@tok@sb\endcsname{\def\PY@tc##1{\textcolor[rgb]{0.73,0.13,0.13}{##1}}}
\expandafter\def\csname PY@tok@sc\endcsname{\def\PY@tc##1{\textcolor[rgb]{0.73,0.13,0.13}{##1}}}
\expandafter\def\csname PY@tok@dl\endcsname{\def\PY@tc##1{\textcolor[rgb]{0.73,0.13,0.13}{##1}}}
\expandafter\def\csname PY@tok@s2\endcsname{\def\PY@tc##1{\textcolor[rgb]{0.73,0.13,0.13}{##1}}}
\expandafter\def\csname PY@tok@sh\endcsname{\def\PY@tc##1{\textcolor[rgb]{0.73,0.13,0.13}{##1}}}
\expandafter\def\csname PY@tok@s1\endcsname{\def\PY@tc##1{\textcolor[rgb]{0.73,0.13,0.13}{##1}}}
\expandafter\def\csname PY@tok@mb\endcsname{\def\PY@tc##1{\textcolor[rgb]{0.40,0.40,0.40}{##1}}}
\expandafter\def\csname PY@tok@mf\endcsname{\def\PY@tc##1{\textcolor[rgb]{0.40,0.40,0.40}{##1}}}
\expandafter\def\csname PY@tok@mh\endcsname{\def\PY@tc##1{\textcolor[rgb]{0.40,0.40,0.40}{##1}}}
\expandafter\def\csname PY@tok@mi\endcsname{\def\PY@tc##1{\textcolor[rgb]{0.40,0.40,0.40}{##1}}}
\expandafter\def\csname PY@tok@il\endcsname{\def\PY@tc##1{\textcolor[rgb]{0.40,0.40,0.40}{##1}}}
\expandafter\def\csname PY@tok@mo\endcsname{\def\PY@tc##1{\textcolor[rgb]{0.40,0.40,0.40}{##1}}}
\expandafter\def\csname PY@tok@ch\endcsname{\let\PY@it=\textit\def\PY@tc##1{\textcolor[rgb]{0.25,0.50,0.50}{##1}}}
\expandafter\def\csname PY@tok@cm\endcsname{\let\PY@it=\textit\def\PY@tc##1{\textcolor[rgb]{0.25,0.50,0.50}{##1}}}
\expandafter\def\csname PY@tok@cpf\endcsname{\let\PY@it=\textit\def\PY@tc##1{\textcolor[rgb]{0.25,0.50,0.50}{##1}}}
\expandafter\def\csname PY@tok@c1\endcsname{\let\PY@it=\textit\def\PY@tc##1{\textcolor[rgb]{0.25,0.50,0.50}{##1}}}
\expandafter\def\csname PY@tok@cs\endcsname{\let\PY@it=\textit\def\PY@tc##1{\textcolor[rgb]{0.25,0.50,0.50}{##1}}}

\def\PYZbs{\char`\\}
\def\PYZus{\char`\_}
\def\PYZob{\char`\{}
\def\PYZcb{\char`\}}
\def\PYZca{\char`\^}
\def\PYZam{\char`\&}
\def\PYZlt{\char`\<}
\def\PYZgt{\char`\>}
\def\PYZsh{\char`\#}
\def\PYZpc{\char`\%}
\def\PYZdl{\char`\$}
\def\PYZhy{\char`\-}
\def\PYZsq{\char`\'}
\def\PYZdq{\char`\"}
\def\PYZti{\char`\~}
% for compatibility with earlier versions
\def\PYZat{@}
\def\PYZlb{[}
\def\PYZrb{]}
\makeatother


    % Exact colors from NB
    \definecolor{incolor}{rgb}{0.0, 0.0, 0.5}
    \definecolor{outcolor}{rgb}{0.545, 0.0, 0.0}



    
    % Prevent overflowing lines due to hard-to-break entities
    \sloppy 
    % Setup hyperref package
    \hypersetup{
      breaklinks=true,  % so long urls are correctly broken across lines
      colorlinks=true,
      urlcolor=urlcolor,
      linkcolor=linkcolor,
      citecolor=citecolor,
      }
    % Slightly bigger margins than the latex defaults
    
    \geometry{verbose,tmargin=1in,bmargin=1in,lmargin=1in,rmargin=1in}
    
    

    \begin{document}
     \date{Miércoles 2 de Junio}     
    
    \maketitle
    
    

    
%    \hypertarget{guuxeda-7-muxe9todos-numuxe9ricos-2021}{%
%\section*{Guía 7 Métodos Numéricos
%2021}\label{guuxeda-7-muxe9todos-numuxe9ricos-2021}}

%    \begin{Verbatim}[commandchars=\\\{\}]
%{\color{incolor}In [{\color{incolor} }]:} \PY{k}{using} \PY{n}{Plots}
%\end{Verbatim}


    \hypertarget{problema-1}{%
\subsection*{Problema 1}\label{problema-1}}

Escriba funciones que permitan realizar un paso de integración de la
ecuación,
\[
\frac{dy}{dt} = f(t,y,p),\;\;\;\;\;\; a\leq t \leq b,\;\;\;\;\; y(a) = \alpha
\]
de una función arbitraria \(f(t,x,p)\) de acuerdo a los métodos de
Euler, y Runge Kutta de 2° orden (RK2) y Runge Kutta de 4° orden (RK4).
Las funciones deben admitir como variables: \(f(t,x,p)\), \(x_0\) (el
valor inicial), \(t_0\) y \(h\) (el paso de integración), el campo \(p\)
es para permitir parámetros en la función. Cada función debe retornar el
valor de la variable luego del paso de integración, es decir la
aproximación a \(x(t + h)\).

Utilizando estas funciones escriba otra función que permita hacer,
tomando como variable cualquiera de los métodos, la integración de una
función genérica en un intervalo \([a,b]\) arbitrario. Esta función
deberá admitir como variables, además de las anteriores, la función de
un paso de cada método, y el intervalo de integración. La función debe
retornar dos vectores, uno con los valores \(t_i = t_0 + i*h\) y otro
con los valores aproximados de \(x_i \approx x(t_i)\).

\textbf{Ayuda:} Dejamos como ejemplo la implementación del método de
Euler.

    \begin{Verbatim}[commandchars=\\\{\}]
{\color{incolor}In [{\color{incolor} }]:} \PY{l+s}{\PYZdq{}\PYZdq{}\PYZdq{}}
        \PY{l+s}{ }\PY{l+s}{ }\PY{l+s}{ }\PY{l+s}{ }\PY{l+s}{E}\PY{l+s}{u}\PY{l+s}{l}\PY{l+s}{e}\PY{l+s}{r}\PY{l+s}{(}\PY{l+s}{f}\PY{l+s}{,}\PY{l+s}{y}\PY{l+s}{0}\PY{l+s}{,}\PY{l+s}{t}\PY{l+s}{0}\PY{l+s}{,}\PY{l+s}{h}\PY{l+s}{,}\PY{l+s}{p}\PY{l+s}{)}
        
        \PY{l+s}{H}\PY{l+s}{a}\PY{l+s}{c}\PY{l+s}{e}\PY{l+s}{ }\PY{l+s}{u}\PY{l+s}{n}\PY{l+s}{ }\PY{l+s}{p}\PY{l+s}{a}\PY{l+s}{s}\PY{l+s}{o}\PY{l+s}{ }\PY{l+s}{d}\PY{l+s}{e}\PY{l+s}{l}\PY{l+s}{ }\PY{l+s}{m}\PY{l+s}{é}\PY{l+s}{t}\PY{l+s}{o}\PY{l+s}{d}\PY{l+s}{o}\PY{l+s}{ }\PY{l+s}{d}\PY{l+s}{e}\PY{l+s}{ }\PY{l+s}{E}\PY{l+s}{u}\PY{l+s}{l}\PY{l+s}{e}\PY{l+s}{r}\PY{l+s}{ }\PY{l+s}{e}\PY{l+s}{x}\PY{l+s}{p}\PY{l+s}{l}\PY{l+s}{í}\PY{l+s}{c}\PY{l+s}{i}\PY{l+s}{t}\PY{l+s}{o}\PY{l+s}{:}\PY{l+s}{ }
        \PY{l+s}{ }\PY{l+s}{ }\PY{l+s}{ }\PY{l+s}{ }\PY{l+s}{f}\PY{l+s}{ }\PY{l+s}{=}\PY{l+s}{ }\PY{l+s}{f}\PY{l+s}{u}\PY{l+s}{n}\PY{l+s}{c}\PY{l+s}{i}\PY{l+s}{ó}\PY{l+s}{n}\PY{l+s}{ }\PY{l+s}{q}\PY{l+s}{u}\PY{l+s}{e}\PY{l+s}{ }\PY{l+s}{n}\PY{l+s}{o}\PY{l+s}{s}\PY{l+s}{ }\PY{l+s}{d}\PY{l+s}{a}\PY{l+s}{ }\PY{l+s}{l}\PY{l+s}{a}\PY{l+s}{ }\PY{l+s}{t}\PY{l+s}{a}\PY{l+s}{n}\PY{l+s}{g}\PY{l+s}{e}\PY{l+s}{n}\PY{l+s}{t}\PY{l+s}{e}\PY{l+s}{ }\PY{l+s}{c}\PY{l+s}{o}\PY{l+s}{m}\PY{l+s}{o}\PY{l+s}{ }\PY{l+s}{e}\PY{l+s}{n}\PY{l+s}{ }\PY{l+s}{(}\PY{l+s}{y}\PY{l+s}{,}\PY{l+s}{t}\PY{l+s}{,}\PY{l+s}{p}\PY{l+s}{)}
        \PY{l+s}{ }\PY{l+s}{ }\PY{l+s}{ }\PY{l+s}{ }\PY{l+s}{y}\PY{l+s}{0}\PY{l+s}{ }\PY{l+s}{=}\PY{l+s}{ }\PY{l+s}{y}\PY{l+s}{ }\PY{l+s}{i}\PY{l+s}{n}\PY{l+s}{i}\PY{l+s}{c}\PY{l+s}{i}\PY{l+s}{a}\PY{l+s}{l}
        \PY{l+s}{ }\PY{l+s}{ }\PY{l+s}{ }\PY{l+s}{ }\PY{l+s}{t}\PY{l+s}{0}\PY{l+s}{ }\PY{l+s}{=}\PY{l+s}{ }\PY{l+s}{t}\PY{l+s}{ }\PY{l+s}{i}\PY{l+s}{n}\PY{l+s}{i}\PY{l+s}{c}\PY{l+s}{i}\PY{l+s}{a}\PY{l+s}{l}
        \PY{l+s}{ }\PY{l+s}{ }\PY{l+s}{ }\PY{l+s}{ }\PY{l+s}{h}\PY{l+s}{ }\PY{l+s}{=}\PY{l+s}{ }\PY{l+s}{d}\PY{l+s}{t}
        \PY{l+s}{ }\PY{l+s}{ }\PY{l+s}{ }\PY{l+s}{ }\PY{l+s}{p}\PY{l+s}{ }\PY{l+s}{=}\PY{l+s}{ }\PY{l+s}{p}\PY{l+s}{a}\PY{l+s}{r}\PY{l+s}{a}\PY{l+s}{m}\PY{l+s}{e}\PY{l+s}{t}\PY{l+s}{r}\PY{l+s}{o}\PY{l+s}{s}\PY{l+s}{ }\PY{l+s}{o}\PY{l+s}{p}\PY{l+s}{c}\PY{l+s}{i}\PY{l+s}{o}\PY{l+s}{n}\PY{l+s}{a}\PY{l+s}{l}\PY{l+s}{e}\PY{l+s}{s}\PY{l+s}{.}
        
        \PY{l+s}{\PYZsh{}}\PY{l+s}{ }\PY{l+s}{E}\PY{l+s}{x}\PY{l+s}{a}\PY{l+s}{m}\PY{l+s}{p}\PY{l+s}{l}\PY{l+s}{e}\PY{l+s}{s}
        \PY{l+s}{`}\PY{l+s}{`}\PY{l+s}{`}\PY{l+s}{j}\PY{l+s}{u}\PY{l+s}{l}\PY{l+s}{i}\PY{l+s}{a}\PY{l+s}{\PYZhy{}}\PY{l+s}{r}\PY{l+s}{e}\PY{l+s}{p}\PY{l+s}{l}
        \PY{l+s}{j}\PY{l+s}{u}\PY{l+s}{l}\PY{l+s}{i}\PY{l+s}{a}\PY{l+s}{\PYZgt{}}\PY{l+s}{ }
        \PY{l+s}{f}\PY{l+s}{u}\PY{l+s}{n}\PY{l+s}{c}\PY{l+s}{t}\PY{l+s}{i}\PY{l+s}{o}\PY{l+s}{n}\PY{l+s}{ }\PY{l+s}{f}\PY{l+s}{(}\PY{l+s}{y}\PY{l+s}{,}\PY{l+s}{t}\PY{l+s}{,}\PY{l+s}{p}\PY{l+s}{)}
        \PY{l+s}{ }\PY{l+s}{ }\PY{l+s}{ }\PY{l+s}{ }\PY{l+s}{r}\PY{l+s}{e}\PY{l+s}{t}\PY{l+s}{u}\PY{l+s}{r}\PY{l+s}{n}\PY{l+s}{ }\PY{l+s}{\PYZhy{}}\PY{l+s}{p}\PY{l+s}{[}\PY{l+s}{1}\PY{l+s}{]}\PY{l+s}{*}\PY{l+s}{y}\PY{l+s}{ }\PY{l+s}{+}\PY{l+s}{ }\PY{l+s}{s}\PY{l+s}{i}\PY{l+s}{n}\PY{l+s}{(}\PY{l+s}{2}\PY{l+s}{π}\PY{l+s}{*}\PY{l+s}{t}\PY{l+s}{)}\PY{l+s}{ }\PY{l+s}{+}\PY{l+s}{ }\PY{l+s}{p}\PY{l+s}{[}\PY{l+s}{2}\PY{l+s}{]}
        \PY{l+s}{e}\PY{l+s}{n}\PY{l+s}{d}
        \PY{l+s}{h}\PY{l+s}{=}\PY{l+s}{ }\PY{l+s}{0}\PY{l+s}{.}\PY{l+s}{1}
        \PY{l+s}{E}\PY{l+s}{u}\PY{l+s}{l}\PY{l+s}{e}\PY{l+s}{r}\PY{l+s}{(}\PY{l+s}{f}\PY{l+s}{,}\PY{l+s}{1}\PY{l+s}{,}\PY{l+s}{0}\PY{l+s}{,}\PY{l+s}{h}\PY{l+s}{,}\PY{l+s}{[}\PY{l+s}{1}\PY{l+s}{,}\PY{l+s}{2}\PY{l+s}{]}\PY{l+s}{)}
        \PY{l+s}{1}\PY{l+s}{.}\PY{l+s}{1}
        \PY{l+s}{`}\PY{l+s}{`}\PY{l+s}{`}
        \PY{l+s}{\PYZdq{}\PYZdq{}\PYZdq{}}
        \PY{k}{function} \PY{n}{Euler}\PY{p}{(}\PY{n}{f}\PY{p}{,}\PY{n}{y0}\PY{p}{,}\PY{n}{t0}\PY{p}{,}\PY{n}{h}\PY{p}{,}\PY{n}{p}\PY{p}{)}
            \PY{k}{return} \PY{n}{y0} \PY{o}{+} \PY{n}{h}\PY{o}{*}\PY{n}{f}\PY{p}{(}\PY{n}{y0}\PY{p}{,}\PY{n}{t0}\PY{p}{,}\PY{n}{p}\PY{p}{)}
        \PY{k}{end}
\end{Verbatim}


    \begin{Verbatim}[commandchars=\\\{\}]
{\color{incolor}In [{\color{incolor} }]:} \PY{l+s}{\PYZdq{}\PYZdq{}\PYZdq{}}
        \PY{l+s}{ }\PY{l+s}{ }\PY{l+s}{ }\PY{l+s}{ }\PY{l+s}{O}\PY{l+s}{D}\PY{l+s}{E}\PY{l+s}{p}\PY{l+s}{r}\PY{l+s}{o}\PY{l+s}{b}\PY{l+s}{l}\PY{l+s}{e}\PY{l+s}{m}\PY{l+s}{(}\PY{l+s}{M}\PY{l+s}{e}\PY{l+s}{t}\PY{l+s}{h}\PY{l+s}{o}\PY{l+s}{d}\PY{l+s}{,}\PY{l+s}{f}\PY{l+s}{,}\PY{l+s}{y}\PY{l+s}{0}\PY{l+s}{,}\PY{l+s}{i}\PY{l+s}{n}\PY{l+s}{t}\PY{l+s}{e}\PY{l+s}{r}\PY{l+s}{v}\PY{l+s}{a}\PY{l+s}{l}\PY{l+s}{o}\PY{l+s}{,}\PY{l+s}{N}\PY{l+s}{,}\PY{l+s}{p}\PY{l+s}{)}\PY{l+s}{ }
        
        \PY{l+s}{H}\PY{l+s}{a}\PY{l+s}{c}\PY{l+s}{e}\PY{l+s}{ }\PY{l+s}{u}\PY{l+s}{n}\PY{l+s}{a}\PY{l+s}{ }\PY{l+s}{e}\PY{l+s}{v}\PY{l+s}{o}\PY{l+s}{l}\PY{l+s}{u}\PY{l+s}{c}\PY{l+s}{i}\PY{l+s}{ó}\PY{l+s}{n}\PY{l+s}{ }\PY{l+s}{d}\PY{l+s}{e}\PY{l+s}{ }\PY{l+s}{u}\PY{l+s}{n}\PY{l+s}{a}\PY{l+s}{ }\PY{l+s}{e}\PY{l+s}{c}\PY{l+s}{u}\PY{l+s}{a}\PY{l+s}{c}\PY{l+s}{i}\PY{l+s}{ó}\PY{l+s}{n}\PY{l+s}{ }\PY{l+s}{o}\PY{l+s}{r}\PY{l+s}{d}\PY{l+s}{i}\PY{l+s}{n}\PY{l+s}{d}\PY{l+s}{a}\PY{l+s}{r}\PY{l+s}{i}\PY{l+s}{a}\PY{l+s}{ }\PY{l+s}{d}\PY{l+s}{a}\PY{l+s}{d}\PY{l+s}{a}\PY{l+s}{ }\PY{l+s}{p}\PY{l+s}{o}\PY{l+s}{r}\PY{l+s}{ }\PY{l+s}{f}\PY{l+s}{(}\PY{l+s}{y}\PY{l+s}{,}\PY{l+s}{t}\PY{l+s}{)}\PY{l+s}{ }\PY{l+s}{u}\PY{l+s}{s}\PY{l+s}{a}\PY{l+s}{n}\PY{l+s}{d}\PY{l+s}{o}\PY{l+s}{ }\PY{l+s}{M}\PY{l+s}{e}\PY{l+s}{t}\PY{l+s}{h}\PY{l+s}{o}\PY{l+s}{d}\PY{l+s}{.}
        \PY{l+s}{ }\PY{l+s}{ }\PY{l+s}{ }\PY{l+s}{ }\PY{l+s}{M}\PY{l+s}{e}\PY{l+s}{t}\PY{l+s}{h}\PY{l+s}{o}\PY{l+s}{d}\PY{l+s}{:}\PY{l+s}{ }\PY{l+s}{a}\PY{l+s}{l}\PY{l+s}{g}\PY{l+s}{ú}\PY{l+s}{n}\PY{l+s}{ }\PY{l+s}{m}\PY{l+s}{é}\PY{l+s}{t}\PY{l+s}{o}\PY{l+s}{d}\PY{l+s}{o}\PY{l+s}{ }\PY{l+s}{d}\PY{l+s}{e}\PY{l+s}{ }\PY{l+s}{i}\PY{l+s}{n}\PY{l+s}{t}\PY{l+s}{e}\PY{l+s}{g}\PY{l+s}{r}\PY{l+s}{a}\PY{l+s}{c}\PY{l+s}{i}\PY{l+s}{ó}\PY{l+s}{n}\PY{l+s}{,}\PY{l+s}{ }\PY{l+s}{p}\PY{l+s}{o}\PY{l+s}{r}\PY{l+s}{ }\PY{l+s}{e}\PY{l+s}{j}\PY{l+s}{e}\PY{l+s}{m}\PY{l+s}{p}\PY{l+s}{l}\PY{l+s}{o}\PY{l+s}{ }\PY{l+s}{E}\PY{l+s}{u}\PY{l+s}{l}\PY{l+s}{e}\PY{l+s}{r}\PY{l+s}{(}\PY{l+s}{f}\PY{l+s}{,}\PY{l+s}{y}\PY{l+s}{0}\PY{l+s}{,}\PY{l+s}{t}\PY{l+s}{0}\PY{l+s}{,}\PY{l+s}{h}\PY{l+s}{)}
        \PY{l+s}{ }\PY{l+s}{ }\PY{l+s}{ }\PY{l+s}{ }\PY{l+s}{f}\PY{l+s}{ }\PY{l+s}{=}\PY{l+s}{ }\PY{l+s}{f}\PY{l+s}{u}\PY{l+s}{n}\PY{l+s}{c}\PY{l+s}{i}\PY{l+s}{ó}\PY{l+s}{n}\PY{l+s}{ }\PY{l+s}{q}\PY{l+s}{u}\PY{l+s}{e}\PY{l+s}{ }\PY{l+s}{n}\PY{l+s}{o}\PY{l+s}{s}\PY{l+s}{ }\PY{l+s}{d}\PY{l+s}{a}\PY{l+s}{ }\PY{l+s}{l}\PY{l+s}{a}\PY{l+s}{ }\PY{l+s}{t}\PY{l+s}{a}\PY{l+s}{n}\PY{l+s}{g}\PY{l+s}{e}\PY{l+s}{n}\PY{l+s}{t}\PY{l+s}{e}\PY{l+s}{ }\PY{l+s}{c}\PY{l+s}{o}\PY{l+s}{m}\PY{l+s}{o}\PY{l+s}{ }\PY{l+s}{e}\PY{l+s}{n}\PY{l+s}{ }\PY{l+s}{(}\PY{l+s}{y}\PY{l+s}{,}\PY{l+s}{t}\PY{l+s}{)}
        \PY{l+s}{ }\PY{l+s}{ }\PY{l+s}{ }\PY{l+s}{ }\PY{l+s}{y}\PY{l+s}{0}\PY{l+s}{ }\PY{l+s}{=}\PY{l+s}{ }\PY{l+s}{y}\PY{l+s}{ }\PY{l+s}{i}\PY{l+s}{n}\PY{l+s}{i}\PY{l+s}{c}\PY{l+s}{i}\PY{l+s}{a}\PY{l+s}{l}
        \PY{l+s}{ }\PY{l+s}{ }\PY{l+s}{ }\PY{l+s}{ }\PY{l+s}{i}\PY{l+s}{n}\PY{l+s}{t}\PY{l+s}{e}\PY{l+s}{r}\PY{l+s}{v}\PY{l+s}{a}\PY{l+s}{l}\PY{l+s}{o}\PY{l+s}{ }\PY{l+s}{=}\PY{l+s}{ }\PY{l+s}{(}\PY{l+s}{t}\PY{l+s}{\PYZus{}}\PY{l+s}{i}\PY{l+s}{n}\PY{l+s}{i}\PY{l+s}{c}\PY{l+s}{i}\PY{l+s}{a}\PY{l+s}{l}\PY{l+s}{,}\PY{l+s}{ }\PY{l+s}{t}\PY{l+s}{\PYZus{}}\PY{l+s}{f}\PY{l+s}{i}\PY{l+s}{n}\PY{l+s}{a}\PY{l+s}{l}\PY{l+s}{)}
        \PY{l+s}{ }\PY{l+s}{ }\PY{l+s}{ }\PY{l+s}{ }\PY{l+s}{N}\PY{l+s}{ }\PY{l+s}{=}\PY{l+s}{ }\PY{l+s}{n}\PY{l+s}{ú}\PY{l+s}{m}\PY{l+s}{e}\PY{l+s}{r}\PY{l+s}{o}\PY{l+s}{ }\PY{l+s}{d}\PY{l+s}{e}\PY{l+s}{ }\PY{l+s}{p}\PY{l+s}{a}\PY{l+s}{s}\PY{l+s}{o}\PY{l+s}{s}
        \PY{l+s}{ }\PY{l+s}{ }\PY{l+s}{ }\PY{l+s}{ }\PY{l+s}{p}\PY{l+s}{ }\PY{l+s}{=}\PY{l+s}{ }\PY{l+s}{p}\PY{l+s}{a}\PY{l+s}{r}\PY{l+s}{á}\PY{l+s}{m}\PY{l+s}{e}\PY{l+s}{t}\PY{l+s}{r}\PY{l+s}{o}\PY{l+s}{s}\PY{l+s}{ }\PY{l+s}{o}\PY{l+s}{p}\PY{l+s}{c}\PY{l+s}{i}\PY{l+s}{o}\PY{l+s}{n}\PY{l+s}{a}\PY{l+s}{l}\PY{l+s}{e}\PY{l+s}{s}
        
        \PY{l+s}{\PYZsh{}}\PY{l+s}{ }\PY{l+s}{E}\PY{l+s}{x}\PY{l+s}{a}\PY{l+s}{m}\PY{l+s}{p}\PY{l+s}{l}\PY{l+s}{e}\PY{l+s}{s}
        \PY{l+s}{`}\PY{l+s}{`}\PY{l+s}{`}\PY{l+s}{j}\PY{l+s}{u}\PY{l+s}{l}\PY{l+s}{i}\PY{l+s}{a}\PY{l+s}{\PYZhy{}}\PY{l+s}{r}\PY{l+s}{e}\PY{l+s}{p}\PY{l+s}{l}
        \PY{l+s}{j}\PY{l+s}{u}\PY{l+s}{l}\PY{l+s}{i}\PY{l+s}{a}\PY{l+s}{\PYZgt{}}\PY{l+s}{ }
        \PY{l+s}{f}\PY{l+s}{u}\PY{l+s}{n}\PY{l+s}{c}\PY{l+s}{t}\PY{l+s}{i}\PY{l+s}{o}\PY{l+s}{n}\PY{l+s}{ }\PY{l+s}{f}\PY{l+s}{(}\PY{l+s}{y}\PY{l+s}{,}\PY{l+s}{t}\PY{l+s}{,}\PY{l+s}{p}\PY{l+s}{)}
        \PY{l+s}{ }\PY{l+s}{ }\PY{l+s}{ }\PY{l+s}{ }\PY{l+s}{r}\PY{l+s}{e}\PY{l+s}{t}\PY{l+s}{u}\PY{l+s}{r}\PY{l+s}{n}\PY{l+s}{ }\PY{l+s}{\PYZhy{}}\PY{l+s}{p}\PY{l+s}{[}\PY{l+s}{1}\PY{l+s}{]}\PY{l+s}{*}\PY{l+s}{y}\PY{l+s}{ }\PY{l+s}{+}\PY{l+s}{ }\PY{l+s}{s}\PY{l+s}{i}\PY{l+s}{n}\PY{l+s}{(}\PY{l+s}{2}\PY{l+s}{π}\PY{l+s}{*}\PY{l+s}{t}\PY{l+s}{)}\PY{l+s}{ }\PY{l+s}{+}\PY{l+s}{ }\PY{l+s}{p}\PY{l+s}{[}\PY{l+s}{2}\PY{l+s}{]}
        \PY{l+s}{e}\PY{l+s}{n}\PY{l+s}{d}
        \PY{l+s}{y}\PY{l+s}{0}\PY{l+s}{ }\PY{l+s}{=}\PY{l+s}{ }\PY{l+s}{0}\PY{l+s}{.}\PY{l+s}{0}
        \PY{l+s}{i}\PY{l+s}{n}\PY{l+s}{t}\PY{l+s}{e}\PY{l+s}{r}\PY{l+s}{v}\PY{l+s}{a}\PY{l+s}{l}\PY{l+s}{o}\PY{l+s}{ }\PY{l+s}{=}\PY{l+s}{ }\PY{l+s}{(}\PY{l+s}{0}\PY{l+s}{,}\PY{l+s}{1}\PY{l+s}{)}
        \PY{l+s}{t}\PY{l+s}{,}\PY{l+s}{y}\PY{l+s}{ }\PY{l+s}{=}\PY{l+s}{ }\PY{l+s}{O}\PY{l+s}{D}\PY{l+s}{E}\PY{l+s}{p}\PY{l+s}{r}\PY{l+s}{o}\PY{l+s}{b}\PY{l+s}{l}\PY{l+s}{e}\PY{l+s}{m}\PY{l+s}{(}\PY{l+s}{E}\PY{l+s}{u}\PY{l+s}{l}\PY{l+s}{e}\PY{l+s}{r}\PY{l+s}{,}\PY{l+s}{f}\PY{l+s}{,}\PY{l+s}{y}\PY{l+s}{0}\PY{l+s}{,}\PY{l+s}{i}\PY{l+s}{n}\PY{l+s}{t}\PY{l+s}{e}\PY{l+s}{r}\PY{l+s}{v}\PY{l+s}{a}\PY{l+s}{l}\PY{l+s}{o}\PY{l+s}{,}\PY{l+s}{1}\PY{l+s}{0}\PY{l+s}{1}\PY{l+s}{,}\PY{l+s}{[}\PY{l+s}{1}\PY{l+s}{,}\PY{l+s}{2}\PY{l+s}{]}\PY{l+s}{)}
        \PY{l+s}{`}\PY{l+s}{`}\PY{l+s}{`}
        \PY{l+s}{\PYZdq{}\PYZdq{}\PYZdq{}}
        \PY{k}{function} \PY{n}{ODEproblem}\PY{p}{(}\PY{k+kt}{Method}\PY{p}{,}\PY{n}{f}\PY{p}{,}\PY{n}{y0}\PY{p}{,}\PY{n}{intervalo}\PY{p}{,}\PY{n}{N}\PY{p}{,}\PY{n}{p}\PY{p}{)}
            \PY{n}{a}\PY{p}{,}\PY{n}{b} \PY{o}{=} \PY{n}{intervalo}
            \PY{n}{h} \PY{o}{=} \PY{p}{(}\PY{n}{b}\PY{o}{\PYZhy{}}\PY{n}{a}\PY{p}{)}\PY{o}{/}\PY{p}{(}\PY{n}{N}\PY{o}{\PYZhy{}}\PY{l+m+mi}{1}\PY{p}{)}
            \PY{n}{y} \PY{o}{=} \PY{n}{zeros}\PY{p}{(}\PY{n}{N}\PY{p}{)}
            \PY{n}{t} \PY{o}{=} \PY{n}{zeros}\PY{p}{(}\PY{n}{N}\PY{p}{)}
            \PY{n}{y}\PY{p}{[}\PY{l+m+mi}{1}\PY{p}{]} \PY{o}{=} \PY{n}{y0}
            \PY{n}{t}\PY{p}{[}\PY{l+m+mi}{1}\PY{p}{]} \PY{o}{=} \PY{n}{a}
            \PY{k}{for} \PY{n}{i} \PY{k+kp}{in} \PY{l+m+mi}{2}\PY{o}{:}\PY{n}{N}
                \PY{n}{t}\PY{p}{[}\PY{n}{i}\PY{p}{]} \PY{o}{=} \PY{n}{t}\PY{p}{[}\PY{n}{i}\PY{o}{\PYZhy{}}\PY{l+m+mi}{1}\PY{p}{]} \PY{o}{+} \PY{n}{h}
                \PY{n}{y}\PY{p}{[}\PY{n}{i}\PY{p}{]} \PY{o}{=} \PY{k+kt}{Method}\PY{p}{(}\PY{n}{f}\PY{p}{,}\PY{n}{y}\PY{p}{[}\PY{n}{i}\PY{o}{\PYZhy{}}\PY{l+m+mi}{1}\PY{p}{]}\PY{p}{,}\PY{n}{t}\PY{p}{[}\PY{n}{i}\PY{o}{\PYZhy{}}\PY{l+m+mi}{1}\PY{p}{]}\PY{p}{,}\PY{n}{h}\PY{p}{,}\PY{n}{p}\PY{p}{)}
            \PY{k}{end}
            \PY{k}{return} \PY{p}{(}\PY{n}{t}\PY{p}{[}\PY{o}{:}\PY{p}{]} \PY{p}{,}\PY{n}{y}\PY{p}{[}\PY{o}{:}\PY{p}{]}\PY{p}{)}
        \PY{k}{end}
\end{Verbatim}


    \hypertarget{problema-2}{%
\subsection*{Problema 2}\label{problema-2}}

Utilizando las funciones del \textbf{Problema 1} resuelva con los tres
métodos dados en el teórico (Euler, RK2 y RK4) el siguiente problema de
valores iniciales: \[
\frac{dy}{dt} = -y+\sin(2\pi t), \;\;\;\;\;\; 0 \le t \le 1\; , 
\;\;\;\;\; y(0) = 1.0
\] en el intervalo \(0 \le t \le 1\) con un paso de integración
\(h=0.1\).

Grafique tanto la solución obtenida y compare con la exacta:

\[
x_e(t)=\Bigl(1+\frac{2\pi}{1+4\pi^2}\Bigr)e^{-t}+\frac{\sin(2\pi t)-2\pi
    \cos(2\pi t)}{1+4\pi^2},
\]

Grafique el error global, \(\epsilon(t) = |y(t)-y_e(t)|\)

    \hypertarget{problema-3}{%
\section*{Problema 3}\label{problema-3}}

Considere el problema de valor inicial: \[
\frac{dy}{dx} = \sin{(y)},\;\;\;\;\;\; 0\le t\le 20.0, \;\;\;\;\; y(0)=\alpha
\] Resuélvalo para los siguientes valores iniciales \(\alpha_1=0.5\),
\(\alpha_2=2.0\), \(\alpha_3= \pi\), \(\alpha_4=3.6\) \(\alpha_5=5.5\) y
\(\alpha_6=2\pi\), en todos los casos con \(h=0.1\). Para cada valor
inicial genere un archivo de salida como el indicado en el problema 1
(solo para RK4). Luego grafique simultáneamente las seis curvas
aproximadas a las soluciones de los seis problemas de valores iniciales
(no olvide hacer un gráfico de calidad, completo). Analice.

    \hypertarget{problema-4}{%
\section*{Problema 4}\label{problema-4}}

\textbf{Método de Runge-Kutta de orden 4}

Muestre que la elección dada en el teórico para los pesos \(\vec{b}\),
los nodos \(\vec{c}\) y la matriz \(A\) para el método RK4:
\begin{eqnarray}
\vec{b}&=&(1/6,1/3,1/3,1/6) \\
\vec{c}&=&(0,1/2,1/2,1) \\
a_{2,1}&=&1/2 \\
a_{3,2}&=&1/2 \\
a_{4,3}&=&1
\end{eqnarray} conduce a las ecuaciones RK4 ``clásicas'' dadas en clase.

    \hypertarget{problema-5}{%
\section*{Problema 5}\label{problema-5}}

Considere el problema de valores iniciales para la ecuación de la
dinámica de un péndulo simple de longitud \(l\) \[
\frac{d^2\theta}{d t^2} = - \frac{g}{l} \sin{(\theta)}, \quad
\theta(0)=\theta_0, \quad \frac{d\theta}{d t}(0)= \dot{\theta}_0,
\] donde \(g\) es la acelaración de la gravedad. Definiendo
\(u= \dot{\theta}\) esta ecuación de segundo orden se puede escribir
como un sistema de dos ecuaciones de primer orden \begin{eqnarray}
\frac{d\theta}{d t} &=& u \hspace{5cm} (1)\\
\frac{d u}{d t} &=& - \frac{g}{l} \sin{(\theta)}
\end{eqnarray} mientras que las condiciones iniciales transformadas
quedan \((u(0),\theta(0))=(\dot{\theta}_0,\theta_0)\).

Resuelva este sistema de dos ecuaciones diferenciales ordinarias
acopladas con el método RK4 para \(g=10 m/s^2\) y \(l=1 m\).

\begin{enumerate}
\def\labelenumi{\arabic{enumi}.}
\item
  Grafique \(\theta\) vs. \(t,\) para \(0\le t\le 10,\) con las
  siguientes condiciones iniciales: a) \(u(0)=0\) y \(\theta(0)=0.5\) y
  b) \(u(0)=0\) y \(\theta(0)=0.25\).
\item
  Calcule la energía del sistema en cada paso. Para las condiciones del
  inciso anterior grafique la energía vs. \(t\). Analice la conservación
  para distintos valores de \(h\).
\item
  Para las condiciones iniciales \(\theta(0)=\theta_0,\) y \(u(0)=0\),
  las ecuaciones de movimiento del péndulo se pueden aproximar por las
  siguientes: \begin{eqnarray}
  \frac{d\theta}{d t} &=& u \hspace{5cm} (2)\\
  \frac{d u}{d t} &=& - \frac{g}{l} \theta
  \end{eqnarray} cuando \(\theta_0\ll 1\). Las ecuaciones (2) admiten
  solución exacta \(\theta(t) = \theta_0 \cos(\sqrt{10}t)\). Compare la
  solución exacta con aproximaciones numéricas
  \(\theta_{\mathrm{num}}(t)\) de las ecuaciones (1) y (2) obtenidas con
  el método RK4. Para ello, grafique la diferencia
  \(\theta_{\mathrm{num}}(t)-\theta_0 \cos(\sqrt{10}t)\) en
  \(0\le t\le 10\) para los casos \(\theta_0=1\) y \(\theta_0=10^{-2}\).
\end{enumerate}

\textbf{Ayuda:} Note que \(y(t)\in \mathbb{R}^2\) y
\(f(t,y)\in \mathbb{R}^2\) donde \(y=(y_1,y_2)=(\theta,u)\) y
\(f(t,y)=(f_1(t,y),f_2(t,y))\) con \(f_1(t,y)=y_1\) y
\(f_2(t,y)=-\frac{g}{l}\sin(y_1)\).

    \hypertarget{problemas-complementarios}{%
\section*{Problemas Complementarios}\label{problemas-complementarios}}

    \hypertarget{problema-c.1}{%
\subsection*{Problema C.1}\label{problema-c.1}}

Use el método del disparo para resolver los siguientes problemas de
frontera con una tolerancia de \(10^{-5}\). Se da un valor tentativo
inicial de \(h\) y la solución exacta para comparación.

\begin{enumerate}
\def\labelenumi{\arabic{enumi}.}
\item
  \(1\leq t\leq 2\), comience con \(h=0.5\) \[
  \ddot{x}\,=\,-(\dot{x})^2 \,,\;\;\;x(1)=0\;,\;\;x(2)=\ln{(2)} \,.
  \] Solución exacta \(x=\ln{(t})\).
\item
  \(-1\leq t\leq 0\), comience con \(h=0.25\) \[
  \ddot{x}\,=\,2 x^3\,,\;\;\;x(-1)=\frac{1}{2}\;,\;\;x(0)=\frac{1}{3} \,.
  \] Solución exacta \(x=1/(t+3)\).
\item
  \(1\leq t\leq 2\), comience con \(h=0.05\) \[
  \ddot{x}\,=\,\frac{(t\,\dot{x} )^2\,-9 x^2+4 t^6}{t^5},\;\;\;x(1)=0\;,\;\;x(2)=\ln{(256)} \,.
  \] Solución exacta \(x=t^3\,\ln{(t})\).
\end{enumerate}

\textbf{Ayuda:} Utilice el método de la bisección para encontrar la raíz
de \(F\).

    \hypertarget{problema-c.2}{%
\subsection*{Problema C.2}\label{problema-c.2}}

Considere la siguiente ecuación diferencial \[
y^{\prime \prime} = \frac{1}{8} \left( 32 + 2 x^3 - y y' \right)    \qquad \qquad \mbox{para }
1 \le x \le 3
\] 

\begin{enumerate}
\def\labelenumi{\arabic{enumi}.}
\setcounter{enumi}{1}
\item 
  Utilice el método RK4 en el intervalo \(1 \le x\le 3\) para
  resolver esta ecuación con las condiciones iniciales \(y(1) = 17\),
  \(y'(1) = 0\). Encuentre, además \(y'(3)\).
\item
  Repita el inciso anterior, pero con las condiciones iniciales
  \(y(1) = 17\), \(y'(1) = -40\).
\item
  Resuelva la misma ecuación diferencial con las condiciones de borde
  \(y(1) = 17\), \(y' (3) = 0\) en \(N=400\) puntos equiespaciados de
  \(x\in [1,3]\) usando el método de disparo. Para ello, combine el
  método de la bisección de tolerancia \(10^{-10}\) con la información
  de los incisos anteriores. Grafique la solución \(y\) y su derivada
  \(y'\).
\end{enumerate}

    \hypertarget{problema-c.3}{%
\subsection*{Problema C.3}\label{problema-c.3}}

La llamada \textbf{ecuación logística} \[
\frac{dN}{dt}= r\,N \left(1-\frac{N}{K}\right)
\] describe el crecimiento autolimitado de una población dada
(suponiendo que no interactúa con otras especies y que tiene fuentes
limitadas de alimentos). Fue propuesta por Verhulst en 1838 y permite
describir al menos cualitativamente varios fenómenos poblacionales
observados en la naturaleza. En esta ecuación \(N(t)\) es el número de
individuos de la colonia al tiempo \(t\) y \(K\) es una constante
positiva.

Una solución \(N^*\) se dice estacionaria si se satisface que
\(dN^*/dt=0\), y por ende no cambia en el tiempo. Para esta ecuación es
fácil verificar que sólo existen dos soluciones estacionarias:
\(N_1^*=0\) y \(N_2^*=K\).

Determine cuál de las dos soluciones estacionarias es estable y cuál
inestable resolviendo numéricamente la ecuación diferencial con el
método Runge-Kutta de cuarto orden para \(r=2\), \(K=100\), en el
intervalo \(0\le t \le 50\) con \(h=0.1\) y considerando cinco
condiciones iniciales diferentes: a) \(N(0)= 0\), b) \(N(0)=2\), c)
\(N(0)=50\), d) \(N(0)= 120\) y d) \(N(0)=200\). Grafique
simultáneamente las cinco soluciones \(t\) vs. \(N(t)\) en el intevalo
\(0\le t\le 50\) en un gráfico completo.

    \hypertarget{problema-c.3}{%
\subsection*{Problema C.3}\label{problema-c.3}}

El objeto de este problema es familiarizarse con el uso de una librería
para resolver un sistema de ecuaciones en derivadas parciales. Para ello
les pedimos que reproduzca en su notebook el \textbf{ejemplo 2} de esta
página: https://diffeq.sciml.ai/stable/tutorials/ode\_example/ Se trata
del atractor de Lorenz, un sistema que excibe caos y que es una
simplificación \emph{extrema} de un problema de climatología. Luego de
implementarlo, juegue cambiando las condiciones iniciales y/o
parámetros. Cambie los métodos de integración. Esta librería tiene
decenas de distintos métodos.

\textbf{Nota:} Al comienzo tiene que poner:
\texttt{using\ Plots,\ OrdinaryDiffEq}


    % Add a bibliography block to the postdoc
    
    
    
    \end{document}
